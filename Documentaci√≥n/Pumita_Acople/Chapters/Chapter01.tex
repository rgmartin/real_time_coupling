%************************************************
\chapter{Acople estático}\label{ch:estatico}
%************************************************
El acople entre el flujo neutrónico y los parámetros neutrónico - termohidráulicos en \texttt{Pumita} para el caso de un cálculo estático se puede resumir con la siguiente secuencia:


\begin{enumerate}
    \item Las \emph{Secciones Eficaces} se calculan en función de la distribución de materiales y el quemado.
    \item Las \emph{Secciones Eficaces} se modifican de acuerdo a la distribución de los \emph{Parámetros Neutrónico - Temohidráulicos} en el reactor.
    \item La distribución de \emph{Flujo Neutrónico} se calcula iterativamente a partir del conjunto de \emph{Secciones Eficaces} obtenido en el paso 2.
    \item Se calcula la distribución de los \emph{Parámetros Neutrónico - Temohidráulicos} en el reactor, a partir del \emph{Flujo Neutrónico} obtenido en el paso 3 y del cálculo termohidráulico.
    \item Se regresa al paso 1, con la nueva distribución de los \emph{Parámetros Neutrónico - Temohidráulicos} en el reactor.
\end{enumerate}

Este proceso iterativo se repetirá hasta que el flujo neutrónico no varíe significativamente de un paso a otro. Para ello se puede tomar como criterio que el número de iteraciones que requiera el cálculo de flujo sea $1$.

En el presente capítulo se desglosarán los distintos pasos del acople, citando las variables y procedimientos dentro del código fuente de \texttt{Pumita} que lo hacen posible.

\section{Cálculo inicial de las secciones eficaces}
 El cálculo inicial de las secciones eficaces se realiza con el procedimiento \texttt{InterpolateXS} contenido en \texttt{CAREMTRI.pas} (o en \texttt{CAREMHEXA.pas} según el modelo de \texttt{Pumita}). 
 
 
 \begin{enumerate}
 	\item Se llama al procedimiento \texttt{asignar\_materiales}. Aquí se asignan los \emph{números de materiales} a cada uno de los 14 trozos de los 61 canales del modelo del núcleo del reactor. También se asignan las fracciones de inserción de barra para los trozos en los que la barra de control está parcialmente insertada. 
 	
 	\item Para cada uno de los canales y trozos del núcleo del reactor, se realiza una interpolación de las secciones eficaces y los coeficientes de difusión a partir de las tablas de los números de materiales correspondientes. Esta interpolación tiene en cuenta el quemado de cada canal y trozo. Se emplea la función \texttt{InterpolateXS}. Los resultados obtenidos son almacenados en las matrices \texttt{DIF[canal,trozo,grupo]}, \texttt{SIGMABS[canal,trozo,grupo]}, \texttt{NUSGIMAFIS[canal,trozo,grupo]},\texttt{SIGMAFIS[canal,trozo,grupo]} y \texttt{SCATTERING[canal,trozo,grupo,grupo1]}. 	
 	
 \end{enumerate}

Es importante destacar que las secciones eficaces tabuladas en el archivo externo están \emph{referenciadas} a valores centrales de los parámetros neutrónico- termohidráulicos. Por ejemplo, la concentración de xenón para el que fueron cálculadas en general no tiene que ser nula. Lo mismo ocurre con la concentración de Samario y los parámetros termohidráulicos. 

\section{Modificación de Secciones Eficaces según Parámetros Neutrónico-Termohidráulicos}

Dentro del procedimiento \texttt{InterpolateXS} también se realizan modificaciones a las secciones eficaces y coeficientes de difusión según las distribuciones de los parámetros neutrónico- termohidráulicos en el reactor. Ahora bien, estas modificaciones serán realizadas solamente si las variables switch correspondientes están activadas (su valor es \texttt{True}).

\subsection{Modificación por concentración de Xenón}
Si la variable switch \texttt{conXenon} está activada, se realiza una modificación de las secciones eficaces a partir de la distribución de la concentración de xenón en el núcleo:

Se llama al procedimiento \texttt{agregar\_Xenon}, pasándole como argumentos la posición \texttt{N00=241} de tabla de las secciones eficaces del Xenón y la diferencia entre la concentración de Xenón para cada canal y trozo y la concentración de xenón de referencia (valor central): \texttt{DELTA= ConcXe[canal,grupo]- ConcXe0}.

Las secciones eficaces y coeficientes de difusión para cada canal y trozo del reactor, y para cada grupo de energía, son modificadas según las siguientes fórmulas:

\begin{align*}
D &= D_0 + \Delta X * \frac{\partial D}{\partial X}\\	
\Sigma_{abs} &= \Sigma_{abs,0} + \Delta X * \sigma^{Xe}_{abs}\\	
\Sigma_{scatt} &= \Sigma_{scatt,0} + \Delta X * \sigma^{Xe}_{scatt}\\	
\Sigma_{fis} &= \Sigma_{fis,0} + \Delta X * \sigma^{Xe}_{fis}\\
\end{align*}

donde $D_0$,$\Sigma_{abs,0}$, $\Sigma_{scatt,0}$, $\Sigma_{fis,0}$ y $D$,$\Sigma_{abs}$, $\Sigma_{scatt}$, $\Sigma_{fis}$ son los coeficientes de difusión y las secciones eficaces antes y después de la modificación respectivamente; $\Delta X$ es la diferencia de concentración de xenón con respecto al valor de referencia \texttt{ConcXe0} (esta diferencia es pasada como argumento al procedimiento \texttt{agregar\_Xenon}) ; $\frac{\partial D}{\partial X}$, $\sigma^{Xe}_{abs}$, $ \sigma^{Xe}_{scatt}$,$\sigma^{Xe}_{fis}$ son los valores microscópicos de los coeficientes de difusión y las secciones eficaces para el Xenón, que se encuentran tabulados a partir de la posición \texttt{N00=241} en las tablas externas.

De estas expresiones se puede ver que la modificación de las secciones eficaces y los coeficientes de difusión se hace teniendo en cuenta la \emph{dierencia} de concentración de Xenón respecto al valor central \texttt{ConcXe0}, con lo que queda implicado que las secciones eficaces y los coeficientes de difusión antes de la modificación se encontraban referidos a este valor central. 

Es por tanto importante asegurarse de que las tablas de secciones eficaces que se pasen a \texttt{Pumita} se correspondan con valores calculados para una concentración de xenón igual a la que se pasa como \texttt{ConcXe0}.

\subsection{Modificación por concentración de Samario}

De manera equivalente al caso del Xenón, las secciones eficaces y los coeficientes de difusión son modificadas según la diferencia entre la concentración de Samario y el correspondiente valor central \texttt{Delta=ConcSm[canal,trozo]-ConcSamario0} si la variables switch \texttt{conSamario} está activa. 

En este caso se llama al procedimiento \texttt{agregar\_Samario} con los argumentos \texttt{N0=281} y el \texttt{Delta} descrito previamente. Las modificaciones se realizan según las expresiones:

\begin{align*}
D &= D_0 + \Delta Sm * \frac{\partial D}{\partial Sm}\\	
\Sigma_{abs} &= \Sigma_{abs,0} + \Delta Sm * \sigma^{Sm}_{abs}\\	
\Sigma_{scatt} &= \Sigma_{scatt,0} + \Delta Sm * \sigma^{Sm}_{scatt}\\	
\Sigma_{fis} &= \Sigma_{fis,0} + \Delta Sm * \sigma^{Sm}_{fis}\\
\end{align*}

donde los términos tienen significados equivalentes a los descritos para el Xenón, solo que esta vez están referidos al Samario. 

Se observa nuevamente que las modificaciones se hacen respecto al valor de referencia \texttt{ConcSamario0} por lo que las secciones eficaces tabuladas están referidas a este valor. 




\subsection{Modificación por Parámetros Termohidráulicos}




\section{Cálculo del Flujo Neutrónico}

Dentro de \texttt{InterpolateXS}, una vez que las secciones eficaces han sido calculadas acorde a los materiales, quemado y los parámetros neutrónico - termohidráulicos (aegún las variables switch activas), se pasan dichas secciones eficaces a variables con coordenadas del reticulado del reactor (triangular o hexagonal según la versión de \texttt{Pumita}). 

Así, se obtienen los valores requeridos en las variables \texttt{D[iz,i,j,grupo], SIGTOT[iz,i,j,grupo], NUSIGF[iz,i,j,grupo], NUSIGF0[iz,i,j,grupo], SIGSCATT[iz,i,j,grupo,grupo1] } para todas las coordenadas \texttt{iz,i,j} del reticulado.

Luego se llama al procedimiento \texttt{MeshConstants} que determina las constantes necesarias para el cálculo del flujo neutrónico con la ecuación de difusión a partir de las secciones eficaces determinadas anteriormente. Es en este procedimiento donde se tienen en cuenta las condiciones de borde según se emplee o no el método de la matriz de respuesta. 

Luego se llama al procedimiento iteraciones, que efectivamente realiza un cálculo iterativo del flujo neutrónico. Este cálculo emplea los coeficientes determinados con \texttt{MeshConstants} que a su vez dependen de las secciones eficaces y por tanto de los parámetros neutrónicos- termohidráulicos. El cálculo del flujo implica varias iteraciones hasta que se obtenga una convergencia del valor de $k$ (factor de multiplicación). El número de iteraciones necesarios para la convergencia es un indicador de cuán lejos está el flujo neutrónico antes de llamarse a \texttt{iteraciones} del flujo neutrónico calculado.

\section{Cálculo de los Parámetros Neutrónico - Termohidráulicos}

\subsection{Cálculo de la distribución de concentración de Xenón}
Si la variable switch \texttt{conXenon} está activa, se llama al procedimiento \texttt{Xenon} que recibe como argumento el paso de tiempo \texttt{DT} empleado en el cálculo.

Para el caso de un cálculo estático, se emplea \texttt{DT = 0}. En este caso se determinan las concentraciones de Xenón en equilibrio para cada canal y trozo del reactor, a partir del flujo neutrónico obtenido previamente. 

${Xe}_\infty = \frac{y_I *  \sum_g\Sigma_{fis,g}\phi_g + y_{Xe \sum_g }  \Phi_g  \Sigma_{fis}}{\sum_g\Phi_g *\sigma^{Xe}_{abs,g} + \lambda_{Xe}}$

donde ${Xe}_\infty$ es la concentración de Xenón en equilibrio calculada para cada canal,trozo ; $y_{Xe}= 0.002576$ y $y_I=0.062810$ son los yields del Xenón y el Iodo en la fisión neutrónica respectivamente ; $\phi_g$ es el flujo neutrónico por grupo ; $\Sigma_{fis,g}$  la sección eficaz de fisión por grupo calculada previamente, $\lambda_{Xe}=2.10e-5 {seg}^{-1}$ la constante temporal de decaimiento del Xenón, y $\sigma^{Xe}_{abs,g}$ la sección eficaz de aborción del Xenón por grupo, tabulada externamente.

De esta manera, la concentración de xenón depende del flujo neutrónico calculado previamente, y más tarde, a través de las modificaciones correspondientes de las secciones eficaces, influye sobre él, cerrando así el acople.

Dentro del procedimiento \texttt{Xenon} también se calcula la concentración media de Xenón y se almacena en la variable $ConcXeMedia$.

\subsection{Cálculo de la distribución de la concentración de Samario}

De manera parecida al cálculo del xenón, el cálculo de xenón se activa con la variable switch \texttt{conSamario}. En este caso se llama al procedimiento \texttt{Samario}. Si el cálculo es estático, se le pasa como argumento \texttt{DT=0}.

Para el caso estático, se calcula la concentración en equilibrio del samario:

${Sm}_\infty = \frac{y_{Pt} *\sum_g \Sigma_{fis,g}*\Phi_g }{\sum_g\sigma^{Sm}_{abs,g}*\Phi_g }$

donde $y_{Pt}$ es el yield del Platino en la fisión neutrónica, $\Sigma_{fis,g}$ es la sección eficaz de fisión por grupo calculada previamente, $\Phi_g$ el flujo neutronigo por grupo para cada canal y trozo y $\sigma^{Sm}_{abs,g}$ es la sección eficaz de absorción del Samario.

El resultado es almacenado en la matriz \texttt{ConcSm[canal,troz]} para cada canal y trozo del núcleo del reactor. Esta variable es luego empleada en la modificación de las secciones eficaces, cerrando así el acople.

\subsection{Cálculo de la distribución de los parámetros termohidráulicos}

Los parámetros termohidráulicos \emph{Densidad de Refrigerante}, \emph{Temperatura del refrigerante} y \emph{Temperatura del combustible} se encuentran almacenados en las matrices globales \texttt{DensRefr}, \texttt{TempRefr} y \texttt{TComb}. 

Estas variables pueden ser modificadas a través del procedimiento \texttt{pasar\_parametros\_TH} que ofrece \texttt{Pumita} en su interfaz. Por tanto, los valores de la distribución de estos parámetros deberá de ser calculada con un código termohidráulico externo y pasarse los correspondientes valores a \texttt{Pumita} para que el mismo los emplee en la modificación correspondiente de las secciones eficaces. 

El cálculo de los valores termohidráulicos y el posterior trapsaso a \texttt{Pumita} deberá de hacerse de forma alternada con el cálculo del flujo neutrónico para obtener un resultado acoplado. 

En general, el cálculo de los parámetros neutrónico termohidráulicos con un programa externo requerirá un conocimiento de la distribución de potencia neutrónicas en el núcleo, lo cual se puede obtener de \texttt{Pumita} a partir del procedimiento (también perteneciente a la interfaz) \texttt{pedir\_distribucion\_potencias}.




As of version 3.0, \texttt{classicthesis} can also be easily used with
\mLyX\footnote{\url{http://www.lyx.org}} thanks to Nicholas Mariette
and Ivo Pletikosić. The \mLyX\ version of this manual will contain
more information on the details.

This should enable anyone with a basic knowledge of \LaTeXe\ or \mLyX\ to
produce beautiful documents without too much effort. In the end, this
is my overall goal: more beautiful documents, especially theses, as I
am tired of seeing so many ugly ones.

The whole template and the used style is released under the
\acsfont{GNU} General Public License.

If you like the style then I would appreciate a postcard:
\begin{center}
    André Miede \\
    Detmolder Straße 32 \\
    31737 Rinteln \\
    Germany
\end{center}
The postcards I received so far are available at:
\begin{center}
    \url{http://postcards.miede.de}
\end{center}
\marginpar{A well-balanced line width improves the legibility of
the text. That's what typography is all about, right?}
So far, many theses, some books, and several other publications have
been typeset successfully with it. If you are interested in some
typographic details behind it, enjoy Robert Bringhurst's wonderful book.
% \citep{bringhurst:2002}.

\paragraph{Important Note:} Some things of this style might look
unusual at first glance, many people feel so in the beginning.
However, all things are intentionally designed to be as they are,
especially these:
\begin{itemize}
    \item No bold fonts are used. Italics or spaced small caps do the
    job quite well.
    \item The size of the text body is intentionally shaped like it
    is. It supports both legibility and allows a reasonable amount of
    information to be on a page. And, no: the lines are not too short.
    \item The tables intentionally do not use vertical or double
    rules. See the documentation for the \texttt{booktabs} package for
    a nice discussion of this topic.\footnote{To be found online at
    \url{http://mirror.ctan.org/macros/latex/contrib/booktabs/}.}
    \item And last but not least, to provide the reader with a way
    easier access to page numbers in the table of contents, the page
    numbers are right behind the titles. Yes, they are \emph{not}
    neatly aligned at the right side and they are \emph{not} connected
    with dots that help the eye to bridge a distance that is not
    necessary. If you are still not convinced: is your reader
    interested in the page number or does she want to sum the numbers
    up?
\end{itemize}
Therefore, please do not break the beauty of the style by changing
these things unless you really know what you are doing! Please.

\paragraph{Yet Another Important Note:} Since \texttt{classicthesis}'
first release in 2006, many things have changed in the \LaTeX\ world.
Trying to keep up-to-date, \texttt{classicthesis} grew and evolved
into many directions, trying to stay (some kind of) stable and be
compatible with its port to \mLyX. However, there are still many
remains from older times in the code, many dirty workarounds here and
there, and several other things I am absolutely not proud of (for
example my unwise combination of \acsfont{KOMA} and
\texttt{titlesec} etc.).
\graffito{An outlook into the future of \texttt{classicthesis}.}

Currently, I am looking into how to completely re-design and
re-implement \texttt{classicthesis} making it easier to maintain and
to use. As a general idea, \texttt{classicthesis.sty} should be
developed and distributed separately from the template bundle itself.
Excellent spin-offs such as \texttt{arsclassica} could also be
integrated (with permission by their authors) as format configurations.
Also, current trends of \texttt{microtype}, \texttt{fontspec}, etc.
should be included as well. As I am not really into deep
\LaTeX\ programming,
I will reach out to the \LaTeX\ community for their expertise and help.


\section{Organization}
A very important factor for successful thesis writing is the
organization of the material. This template suggests a structure as
the following:
\begin{itemize}
    \marginpar{You can use these margins for summaries of the text
    body\dots}
    \item\texttt{Chapters/} is where all the ``real'' content goes in
    separate files such as \texttt{Chapter01.tex} etc.
    % \item\texttt{Examples/} is where you store all listings and other
    % examples you want to use for your text.
    \item\texttt{FrontBackMatter/} is where all the stuff goes that
    surrounds the ``real'' content, such as the acknowledgments,
    dedication, etc.
    \item\texttt{gfx/} is where you put all the graphics you use in
    the thesis. Maybe they should be organized into subfolders
    depending on the chapter they are used in, if you have a lot of
    graphics.
    \item\texttt{Bibliography.bib}: the Bib\TeX\ database to organize
    all the references you might want to cite.
    \item\texttt{classicthesis.sty}: the style definition to get this
    awesome look and feel. Does not only work with this thesis template
    but also on its own (see folder \texttt{Examples}). Bonus: works
    with both \LaTeX\ and \textsc{pdf}\LaTeX\dots and \mLyX.
    % \item\texttt{ClassicThesis.tcp} a \TeX nicCenter project file.
    Great tool and it's free!
    \item\texttt{ClassicThesis.tex}: the main file of your thesis
    where all gets bundled together.
    \item\texttt{classicthesis-config.tex}: a central place to load all
    nifty packages that are used. % In there, you can also activate
    % backrefs in order to have information in the bibliography about
    % where a source was cited in the text (\ie, the page number).

    \emph{Make your changes and adjustments here.} This means that you
    specify here the options you want to load \texttt{classicthesis.sty}
    with. You also adjust the title of your thesis, your name, and all
    similar information here. Refer to \autoref{sec:custom} for more
    information.

    This had to change as of version 3.0 in order to enable an easy
    transition from the ``basic'' style to \mLyX.
\end{itemize}
In total, this should get you started in no time.


\clearpage
\section{Style Options}\label{sec:options}
There are a couple of options for \texttt{classicthesis.sty} that
allow for a bit of freedom concerning the layout:
\marginpar{\dots or your supervisor might use the margins for some
    comments of her own while reading.}
\begin{itemize}
    \item General:
        \begin{itemize}
            \item\texttt{drafting}: prints the date and time at the bottom of
            each page, so you always know which version you are dealing with.
            Might come in handy not to give your Prof. that old draft.
        \end{itemize}

    \item Parts and Chapters:
        \begin{itemize}
            \item\texttt{parts}: if you use Part divisions for your document,
            you should choose this option. (Cannot be used together with
            \texttt{nochapters}.)

            \item\texttt{linedheaders}: changes the look of the chapter
            headings a bit by adding a horizontal line above the chapter
            title. The chapter number will also be moved to the top of the
            page, above the chapter title.
        \end{itemize}

    \item Typography:
        \begin{itemize}
            \item\texttt{eulerchapternumbers}: use figures from Hermann Zapf's
            Euler math font for the chapter numbers. By default, old style
            figures from the Palatino font are used.

            \item\texttt{beramono}: loads Bera Mono as typewriter font.
            (Default setting is using the standard CM typewriter font.)

            \item\texttt{eulermath}: loads the awesome Euler fonts for math.
            Pala\-tino is used as default font.
        \end{itemize}

    \marginpar{Options are enabled via \texttt{option=true}}

    \item Table of Contents:
        \begin{itemize}
            \item\texttt{tocaligned}: aligns the whole table of contents on
            the left side. Some people like that, some don't.

            \item\texttt{dottedtoc}: sets pagenumbers flushed right in the
            table of contents.

            \item\texttt{manychapters}: if you need more than nine chapters for
            your document, you might not be happy with the spacing between the
            chapter number and the chapter title in the Table of Contents.
            This option allows for additional space in this context.
            However, it does not look as ``perfect'' if you use
            \verb|\parts| for structuring your document.
        \end{itemize}

    \item Floats:
        \begin{itemize}
            \item\texttt{listings}: loads the \texttt{listings} package (if not
            already done) and configures the List of Listings accordingly.

            \item\texttt{floatperchapter}: activates numbering per chapter for
            all floats such as figures, tables, and listings (if used).
        \end{itemize}

\end{itemize}

Furthermore, pre-defined margins for different paper sizes are available, \eg, \texttt{a4paper}, \texttt{a5paper}, and \texttt{letterpaper}. These are based on your chosen option of \verb|\documentclass|.

The best way to figure these options out is to try the different
possibilities and see what you and your supervisor like best.

In order to make things easier, \texttt{classicthesis-config.tex}
contains some useful commands that might help you.


\section{Customization}\label{sec:custom}
%(As of v3.0, the Classic Thesis Style for \LaTeX{} and \mLyX{} share
%the same two \texttt{.sty} files.)
This section will show you some hints how to adapt
\texttt{classicthesis} to your needs.

The file \texttt{classicthesis.sty}
contains the core functionality of the style and in most cases will
be left intact, whereas the file \texttt{classic\-thesis-config.tex}
is used for some common user customizations.

The first customization you are about to make is to alter the document
title, author name, and other thesis details. In order to do this, replace
the data in the following lines of \texttt{classicthesis-config.tex:}%
\marginpar{Modifications in \texttt{classic\-thesis-config.tex}%
}

\begin{lstlisting}
    % **************************************************
    % 2. Personal data and user ad-hoc commands
    % **************************************************
    \newcommand{\myTitle}{A Classic Thesis Style\xspace}
    \newcommand{\mySubtitle}{An Homage to...\xspace}
\end{lstlisting}

Further customization can be made in \texttt{classicthesis-config.tex}
by choosing the options to \texttt{classicthesis.sty}
(see~\autoref{sec:options}) in a line that looks like this:

\begin{lstlisting}
  \PassOptionsToPackage{
    drafting=true,    % print version information on the bottom of the pages
    tocaligned=false, % the left column of the toc will be aligned (no indentation)
    dottedtoc=false,  % page numbers in ToC flushed right
    parts=true,       % use part division
    eulerchapternumbers=true, % use AMS Euler for chapter font (otherwise Palatino)
    linedheaders=false,       % chaper headers will have line above and beneath
    floatperchapter=true,     % numbering per chapter for all floats (i.e., Figure 1.1)
    listings=true,    % load listings package and setup LoL
    subfig=true,      % setup for preloaded subfig package
    eulermath=false,  % use awesome Euler fonts for mathematical formulae (only with pdfLaTeX)
    beramono=true,    % toggle a nice monospaced font (w/ bold)
    minionpro=false   % setup for minion pro font; use minion pro small caps as well (only with pdfLaTeX)
  }{classicthesis}
\end{lstlisting}

Many other customizations in \texttt{classicthesis-config.tex} are
possible, but you should be careful making changes there, since some
changes could cause errors.

% Finally, changes can be made in the file \texttt{classicthesis.sty},%
% \marginpar{Modifications in \texttt{classicthesis.sty}%
% } although this is mostly not designed for user customization. The
% main change that might be made here is the text-block size, for example,
% to get longer lines of text.


\section{Issues}\label{sec:issues}
This section will list some information about problems using
\texttt{classic\-thesis} in general or using it with other packages.

Beta versions of \texttt{classicthesis} can be found at Bitbucket:
\begin{center}
    \url{https://bitbucket.org/amiede/classicthesis/}
\end{center}
There, you can also post serious bugs and problems you encounter.


\section{Future Work}
So far, this is a quite stable version that served a couple of people
well during their thesis time. However, some things are still not as
they should be. Proper documentation in the standard format is still
missing. In the long run, the style should probably be published
separately, with the template bundle being only an application of the
style. Alas, there is no time for that at the moment\dots it could be
a nice task for a small group of \LaTeX nicians.

Please do not send me email with questions concerning \LaTeX\ or the
template, as I do not have time for an answer. But if you have
comments, suggestions, or improvements for the style or the template
in general, do not hesitate to write them on that postcard of yours.


\section{Beyond a Thesis}
The layout of \texttt{classicthesis.sty} can be easily used without the
framework of this template. A few examples where it was used to typeset
an article, a book or a curriculum vitae can be found in the folder
\texttt{Examples}. The examples have been tested with
\texttt{latex} and \texttt{pdflatex} and are easy to compile. To
encourage you even more, PDFs built from the sources can be found in the
same folder.


\section{License}
\paragraph{GNU General Public License:} This program is free software;
you can redistribute it and/or modify
it under the terms of the \acsfont{GNU} General Public License as
published by
the Free Software Foundation; either version 2 of the License, or
(at your option) any later version.

This program is distributed in the hope that it will be useful,
but \emph{without any warranty}; without even the implied warranty of
\emph{merchant\-ability} or \emph{fitness for a particular purpose}.
See the
\acsfont{GNU} General Public License for more details.

You should have received a copy of the \acsfont{GNU} General
Public License
along with this program; see the file \texttt{COPYING}.  If not,
write to
the Free Software Foundation, Inc., 59 Temple Place - Suite 330,
Boston, MA 02111-1307, USA.

\paragraph{classichthesis Authors' note:} There have been some discussions about the GPL's implications on using \texttt{classicthesis} for theses etc. Details can be found here:
\begin{center}
  \url{https://bitbucket.org/amiede/classicthesis/issues/123/}
\end{center}

We chose (and currently stick with) the GPL because we would not like to compete with proprietary modified versions of our own work. However, the whole template is free as free beer and free speech. We will not demand the sources for theses, books, CVs, etc. that were created using \texttt{classicthesis}.

Postcards are still highly appreciated.





%*****************************************
%*****************************************
%*****************************************
%*****************************************
%*****************************************
